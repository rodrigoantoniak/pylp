\normalsize{ \indent
Esta investigación es de caracter experimental,
considerando que se utilizará una base de datos
propia; pudiéndose manipularse según se requiera,
medir los efectos que ocurran con cada inteligencia
artificial y controlar su validez con triggers
y controles a más alto nivel de código. Podrá
medirse el tiempo y el espacio utilizado para
cada una de las inteligencias, teniendo un grado
de manipulación importante; así como grupos
experimentales (cada una de las planificaciones
por inteligencia artificial) y uno de control
(planificación hecha de forma manual).
}
\newline
\normalsize{ \indent
La forma en que se configurará cada grupo
experimental será aplicando equivalencia
inicial a través de emparejamiento, donde
el plan de estudio será el mismo en cada
prueba; se procurará de mantener el mismo
grupo de docentes (aunque sea generado
aleatoriamente) y se comparará con los
grupos que asocien los mismos docentes con
las materias.
}
\newline
\normalsize{ \indent
El tipo de experimento a presentar es
cuasiexperimental, tomando en consideración que
el plan de estudio de cada carrera de la
escuela secundaria a usar permanecerá intacto;
de todas formas, se aclara que la generación
de nombres de docentes será aleatorio y su
asociación con materias será arbitrario.
}
\newline
\normalsize{ \indent
La forma de medir cada una de las variables son:
}
\begin{itemize}
  \item \underline{Tiempo de planificación:}
  duración para obtener una planificación correcta
  para el colegio. Esto es comparable para la
  forma manual de ajustar horarios y cada una de
  las inteligencias artificiales (entre si).
  La precisión de medición será a la orden de
  milisegundos.
  \item \underline{Espacio:} cantidad de memoria
  y almacenamiento utilizados durante la ejecución
  de las inteligencias artificiales. Excluye a
  la forma de planificar manual. Dependiendo del
  tamaño ajustado para la aleatoriedad, se considerará
  las unidades de medida Kilobyte, Megabyte y Gigabyte.
  \item \underline{Tiempo de implementación:}
  lapso entre el inicio de escritura de cada
  inteligencia artificial y la completitud de la
  misma. En principio se considerará una precisión
  de tiempo en horas.
  \item \underline{Recursos humanos:} se considera
  la cantidad de personas que involucra para
  que sea posible la planificación de horarios,
  las implicancias de las personas dentro de la
  organización de la escuela (si es que necesita
  agregar o permite reducir secretarios) y las
  acciones posibles para la parte ejecutiva del
  colegio. Se aplica al comparar la planificación
  manual y las de inteligencia artificial,
  contando la cantidad que afecta en cada
  situación.
\end{itemize}
\ \newline
\normalsize{ \indent
A la hora de seleccionar los datos, se usó una
muestra no probabilística; a causa de usar
planificaciones de estudio reales del Instituto
Línea Cuchila, a la vez que los cursos existentes
y los horarios de desarrollo de clases. Si bien,
se generarán nombres aleatorios para los
docentes (resguardando su identidad); no modifica
el hecho que la información a utilizar es verídica.
}
\newline
\normalsize{ \indent
La fuente de datos provista se encuentra en tres
archivos Excel, donde se colocará un ítem por
archivo para describirlo:
}
\begin{itemize}
  \item Contenido de materias por cada año de
  orientación que se ofrece en el colegio
  secundario.
  \item Horario de ejemplo por docente de la
  escuela secundaria.
  \item Asignación actual de docentes a
  materias a cubrir en el instituto.
\end{itemize}
