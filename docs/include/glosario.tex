\newacronym{unam}{UNaM}{Universidad Nacional de Misiones}
\newacronym{fceqyn}{FCEQyN}{Facultad de Ciencias Exactas, Químicas y Naturales}
\newacronym{lsi}{Lic. en Sist. de Información}{Licenciatura en Sistemas de Información}
\newacronym{pylp}{Paradigmas y Leng. de Prog.}{Paradigmas y Lenguajes de Programación}
\newacronym{gnu}{GNU}{GNU's Not Unix}
\newacronym{gnome}{GNOME}{GNU Network Object Model Environment}
\newacronym{kde}{KDE}{K Desktop Environment}
\newacronym{xfce}{Xfce}{X Free Cholesterol Environment}
\newacronym{lxqt}{LXQt}{Lightweight X11 Qt}
\newacronym{vmm}{virt-manager}{Virtual Machine Manager}
\newacronym{mv}{MV}{Máquina Virtual}
\newacronym{so}{SO}{Sistema Operativo}
\newacronym{cpu}{CPU}{Central Processing Unit}
\newacronym{gdm}{GDM}{GNOME Display Manager}
\newacronym{ghc}{GHC}{GNU Haskell Compiler}
\newacronym{tui}{TUI}{Terminal User Interface}
\newglossaryentry{de}
{
  name={entorno de escritorio},
  plural={entornos de escritorio},
  description={Conjunto de software para ofrecer al usuario de
  una computadora una interacción amigable y cómoda. Es una
  implementación de interfaz gráfica de usuario que ofrece
  facilidades de acceso y configuración, como barras de
  herramientas e integración entre aplicaciones con habilidades
  como arrastrar y soltar}
}
\newglossaryentry{wm}
{
  name={gestor de ventanas},
  plural={gestores de ventanas},
  description={Programa informático que controla la
  ubicación y apariencia de las ventanas bajo un
  sistema de ventanas en una interfaz gráfica de usuario}
}
\newglossaryentry{comp}
{
  name={compositor de ventanas},
  plural={compositores de ventanas},
  description={Componente de la interfaz gráfica de una
  computadora que dibuja las ventanas o sus bordes. Éste
  también controla cómo éstas son mostradas y cómo
  interactúan con otras ventanas y el resto del entorno
  del escritorio}
}
\newglossaryentry{serv}
{
  name={servidor de ventanas},
  plural={servidores de ventanas},
  description={Programa cuya tarea principal es coordinar
  la entrada y la salida de sus clientes hacia y desde el
  resto del sistema operativo, el hardware, y otros. El
  servidor gráfico se comunica con sus clientes con el
  protocolo de servidor gráfico}
}
