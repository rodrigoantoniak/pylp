\section{Hipótesis}
\subsection*{Hipótesis principal}
\normalsize{ \indent
La inteligencia artificial considerablemente reduce
los costos, y aumenta la eficiencia y la satisfacción de
los recursos humanos; al planificar horarios docentes en
una escuela secundaria, frente a la organización de
horas manual.
}
\subsection*{Hipótesis nula}
\normalsize{ \indent
La inteligencia artificial no reduce considerablemente
los costos, o no aumenta la eficiencia y la satisfacción de
los recursos humanos; al planificar horarios docentes en
una escuela secundaria, frente a la organización de horas
manual.
}
\section{Variables}
\subsection*{Eficiencia}
\normalsize{ \indent
Rendimiento que poseen recursos a la hora de la aplicación
de la inteligencia artificial, dentro de esos recursos se
considera el tiempo y el espacio; donde este último hace
referencia a la memoria y el almacenamiento en el sistema.
Se medirá con el porcentaje (proveniente de dividir el
tiempo y el espacio de una técnica de inteligencia artificial
sobre otra, multiplicado por cien); donde al obtener mayor
eficiencia, se beneficia a la forma de planificar que
corresponda el valor.
}
\subsection*{Costo}
\normalsize{ \indent
Cantidad de recursos utilizados para poder llevar a cabo
cierto método de IA, tanto de memoria como de
almacenamiento; involucrando la cantidad de registros
necesarios durante la ejecución, el tipo de dato que
maneja y el hardware ocupado. Esto encierra a la
implementación, los recursos humanos, la organización y
la parte ejecutiva. Se medirá con una unidad de
representación de tamaño binaria, donde la unidad
fundamental será el Byte; de todas formas, deberá
observarse los resultados para poder saber si la métrica
será kilobyte, megabyte, gigabite u otro.
}
\subsection*{Satisfacción}
\normalsize{ \indent
Sensación de conformidad ante la planificación de horarios
docentes. Incluye a cualquier persona que influya dentro
de la escuela secundaria. Se medirá con la escala Likert,
donde se escalará con puntajes bipolares; donde un extremo
tendrá "Muy Satisfactorio" y el otro poseerá "Muy
insatisfactorio".
}
