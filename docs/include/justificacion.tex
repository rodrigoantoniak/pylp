\normalsize{ \indent
En la actualidad, todas las computadoras personales utilizan
una interfaz gráfica para interactuar con el usuario. Más
allá del sistema operativo que se utilice, el entorno visual
que se utilice en el espacio personal es esencial; tomando en
cuenta la facilidad de uso, comodidad y personalización, entre
otros aspectos.
}
\newline
\normalsize{ \indent
Dependiendo de la arquitectura del sistema operativo, podrán
existir distintas opciones para los entornos de escritorio de
la computadora personal. En el caso de Windows \cite{dwm_exe},
el motor gráfico está definido a nivel del kernel (como se
muestra en
\url{https://learn.microsoft.com/es-ar/windows-hardware/drivers/display/images/dpy1.png} y
\url{https://en.wikipedia.org/wiki/Architecture_of_Windows_NT#/media/File:Windows_2000_architecture.svg});
esto limita a que deba utilizarse el entorno gráfico que define
Microsoft para su sistema operativo. En cambio, los sistemas
operativos basados en UNIX \cite{x_wayland} pueden elegir su
propio entorno de escritorio (por lo tanto, su gestor de
ventanas) porque su arquitectura en capas lo permite.
}
\newline
\normalsize{ \indent
A continuación, se mostrará cómo se gestiona el entorno gráfico
en un sistema operativo Linux para computadoras personales (se
aclara esto porque los servidores no cuentan normalmente con
ninguna interfaz gráfica por defecto):
}
\newline
\begin{tikzpicture}[
  cuadro/.style={rectangle, draw=black!50, fill=white!3,
  very thick, minimum size=3mm},
]
  \node[cuadro] (usuario) {Usuario};
  \node[cuadro] (gui) [below=of usuario] {\Gls{de}};
  \node[cuadro] (ventanas) [below=of gui] {\Gls{wm}};
  \node[cuadro] (servidor) [below=of ventanas] {\Gls{serv}};
  \node[cuadro] (compositor) [right=of servidor] {\Gls{comp}};
  \node[cuadro] (kernel) [below=of servidor] {Núcleo};
  \node[cuadro] (hw) [below=of kernel] {Hardware};
  \draw[>=triangle 45, <->] (usuario.south) -- (gui.north);
  \draw[>=triangle 45, <->] (gui.south) -- (ventanas.north);
  \draw[>=triangle 45, <->] (ventanas.south) -- (servidor.north);
  \draw[>=triangle 45, <->] (servidor.south) -- (kernel.north);
  \draw[>=triangle 45, <->] (kernel.south) -- (hw.north);
  \draw[>=triangle 45, <->] (servidor.east) -- (compositor.west);
\end{tikzpicture}
\newline
\normalsize{ \indent
Un ejemplo de entorno gráfico completo para el esquema
anteriormente mostrado es el siguiente:
}
\begin{itemize}
  \setlength\itemsep{1pt}
  \item \Gls{serv}: X.Org.
  \item \Gls{comp}: picom.
  \item \Gls{wm}: xfwm4.
  \item \Gls{de}: \acrshort{xfce}.
\end{itemize}
\ \newline
\normalsize{ \indent
En la actualidad, existe una gran cantidad de entornos
de escritorio para \acrshort{gnu}/Linux; desde
\acrshort{gnome} y \acrshort{kde}, hasta \acrshort{xfce},
\acrshort{lxqt} y otros. Cada uno de ellos se caracteriza
por su aspecto, conjunto de herramientas y gestor de
ventanas; además de características comparables entre sí,
como el consumo eléctrico y de memoria.
}
\newline
\normalsize{ \indent
Más allá de los pros y contras que posee cada uno de
los anteriormente citados, todos comparten la propiedad
de ser muy conducidos al uso del ratón o pad táctil. Sin
embargo, el usuario puede decidir por interactuar
directamente con el gestor de ventanas; en consecuencia,
la cantidad de capas que interviene en la interacción
humano-computadora es menor y se favorece al menor
consumo de recursos.
}
\newline
\normalsize{ \indent
Además, debe considerarse la salud de las manos con el
uso del ratón \cite{desorden_extremidades}; donde existen
estudios que demuestran los efectos negativos del mouse
promedio \cite{dermatosis}. A pesar de la existencia de
ratones ergonómicos \cite{hipotermia_munieca}, estos
últimos son personales; cada uno se adapta a un tamaño
particular de mano, incluyendo el hecho de haber varias
formas ergonómicas y un costo económico elevado.
}
\newline
\normalsize{ \indent
Por lo tanto, se puede concretar que la utilización
de un gestor de ventanas en mosaico es factible por
varias razones; algunas de las cuales son el consumo
de recursos y la disminución del uso del mouse, entre
otros (ciertas motivaciones son más enfocadas para
desarrolladores, como la programación y la
personalización; mientras que otras pueden observarse
como subjetivas, evitándose incluirse para mantener
la objetividad de la justificación).
}
