\normalsize{ \indent
Para instalar todos los componentes necesarios en
Ubuntu 22.04 LTS, se siguieron los siguientes pasos:
}
\begin{itemize}
	\item sudo apt install xmonad libghc-extra-dev
	\textbackslash \\ libghc-xmonad-contrib-dev
	libghc-xmonad-wallpaper-dev \textbackslash \\
	xmobar compton curl rofi scrot stterm git
	gnome-screensaver \textbackslash \\ g++ gcc
	libc6-dev libffi-dev libgmp-dev make	xz-utils
	\textbackslash \\ zlib1g-dev gnupg netbase \\
	Donde:
	\begin{itemize}
		\item xmonad es el ejecutable del gestor
		de ventanas.
		\item libghc-extra-dev provee librerías
		extras para \acrshort{ghc}, necesarias
		en el archivo xmonad.hs modificado.
		\item libghc-xmonad-contrib-dev es la
		librería de XMonad Contrib.
		\item libghc-xmonad-wallpaper-dev provee
		la función de seleccionar aleatoriamente
		el fondo de pantalla desde xmonad.hs.
		\item xmobar es el ejecutable de la barra
		de estado.
		\item compton es el \gls{comp}.
		\item curl servirá para una instrucción
		posterior.
		\item rofi es el lanzador de aplicaciones.
		\item scrot es una herramienta para
		capturar pantallas.
		\item stterm es una terminal minimalista
		que servirá para su ejecución en casos
		distintos a la principal (Termonad).
		\item git servirá para clonar el
		repositorio de una aplicación
		\item gnome-screensaver sirve para
		bloquear la pantalla en un entorno con
		\acrshort{gdm}.
		\item El resto de los paquetes
		corresponde a las dependencias dentro de
		un comando a realizar posteriormente.
	\end{itemize}
	\item curl -sSL https://get.haskellstack.org/
	| sh
	\item sudo apt install gobject-introspection
	libgirepository1.0-dev \textbackslash \\
	libgtk-3-dev libvte-2.91-dev libpcre2-dev \\
	Donde todos los paquetes son dependencias
	para poder construir Termonad.
	\item cd directorio/a/gusto
	\item git clone
	\url{https://github.com/cdepillabout/termonad}
	\item cd termonad/
	\item stack build
	\item stack run
\end{itemize}
\ \newline
\normalsize{ \indent
Posteriormente, para poder configurar de forma
correcta; es necesario recompilar todos los
archivos de Haskell, donde:
}
\begin{itemize}
	\item XMonad se compila con el comando
	``xmonad -{}-recompile'' y se reinicia con
	``xmonad -{}-restart''. Esta última acción
	afecta a XMobar, por lo que no se necesita
	más acciones para la barra de estado.
	\item Termonad se configura realizando
	``stack exec -{}-package termonad -{}-package
	colour -{}- termonad'' desde $\sim$/.local/bin.
	Adicionalmente, se copia el archivo
	$\sim$/.cache/termonad/termonad-linux-x86\textunderscore
	64 dentro del directorio /usr/local/bin
	(puede ser cualquier otro directorio
	dentro del PATH, lo único que se busca
	es evitar añadir el directorio origen
	del archivo)
\end{itemize}
\ \newline
\normalsize{ \indent
Para el caso de Rofi, no habrá necesiad de
ninguna compilación; con actualizar el
archivo $\sim$/.config/rofi/config.rasi es
suficiente, ya que el programa lee esta
configuración cada vez que se ejecuta.
}
\newline
\normalsize{ \indent
Por siguiente, para la aplicación que permite
hacer acciones sobre la sesión actual; se
clona desde el repositorio de Distrotube
\url{https://gitlab.com/dwt1/byebye}, así
se cambia lo necesario y se obtiene el
ejecutable. Dentro de lo modificado, se
encuentra la traducción de las opciones
en español y la adaptación de algunos
comandos para que funcionen en la máquina
virtual.
}
\newline
\normalsize{ \indent
Para construir la aplicación, se utiliza
``stack build'' desde el directorio en donde
se clonó el repositorio; para obtener el
ejecutable en \\
.stack-work/install/unicaCarpetaAqui/hash/version/bin/byebye-exe,
que se copia dentro de /usr/local/bin por
las mismas razones que Termonad.
}
\newline
\normalsize{ \indent
Adicionalmente, se ha ajustado la salida
de audio controlada por defecto desde
alsamixer; para ello, se ha seguido los
siguientes pasos:
}
\begin{itemize}
	\item Listar las fuentes de sonido
	con ``pactl list short sources''.
	\item Seleccionar la fuente de sonido
	correcta con ``pactl set-default-source
	nombreFuenteSonido''.
	\item Listar los disipadores de sonido
	con ``pactl list short sinks''.
	\item Seleccionar el disipador de sonido
	correcta con ``pactl set-default-sink
	nombreDisipadorSonido''.
\end{itemize}
\ \newline
\normalsize{ \indent
Con todos estos pasos realizados, se
consigue el entorno de escritorio deseado
en Ubuntu con XMonad. Para obtener las
configuraciones de cada programa y la
aplicación personalizada, se puede acceder
al siguiente repositorio:
\url{https://github.com/rodrigoantoniak/pylp}
}