\normalsize{ \indent
En este trabajo, se abarcará la implantación del
gestor de ventanas en mosaico XMonad; en conjunto con
una configuración personalizada, la cual pueda satisfacer
la mayoría de necesidades de los usuarios en promedio.
Considerando que muchos usuarios utilizan entornos de
escritorio completos, se contempla las utilidades que
las personas generalmente necesitan a continuación:
}
\begin{itemize}
  \item Menú que permita acceder a las aplicaciones
  deseadas de un menú de inicio.
  \item Barra de estado que indica los parámetros
  actuales del sistema (XMobar).
  \item Accesibilidad hacia el control de los
  periféricos de la computadora, como la salida
  de sonido, WiFi, Bluetooth, almacenamiento
  extraíble, entre otros.
  \item Menú que permita realizar distintas
  acciones con la sesión activa, tales como apagar,
  reiniciar, cerrar sesión suspender, hibernar y
  dormir.
\end{itemize}
\ \newline
\normalsize{ \indent
No se pretende demostrar todas las capacidades que
posee XMonad, las cuales son varias; sino que se
alcanza hasta las funciones básicas requeridas para
la adaptación de un usuario, el cual está acostumbrado
a un entorno de escritorio completo.
}
\newline
\normalsize{ \indent
Para el lector que desee explorar las configuraciones
y personalizar su sistema al máximo, con objetivo de
lograr el entorno más cómodo y productivo para sí
mismo; se encomienda que busque obtener sus metas
personales por su cuenta, ya que los gestores de
ventanas en mosaico dan la posibilidad de ajustar el
escritorio a gusto de cada usuario.
}
\newline
\normalsize{ \indent
Adicionalmente, se utilizará el lanzador de aplicaciones
Rofi; en consecuencia, se podrá ejecutar programas sin
necesidad de personalizarlo en el menú de inicio. También,
se cambiará la terminal por defecto de la distribución
de Linux a seleccionar por Termonad; buscando mostrar las
capacidades del lenguaje de programación Haskell.
}
